%%%%%%%%%%%%%%%%%%%%%%%%%%%%%%%%%%%%%%%%%
% Structured General Purpose Assignment
% LaTeX Template
%
% This template has been downloaded from:
% http://www.latextemplates.com
%
% Original author:
% Ted Pavlic (http://www.tedpavlic.com)
%
% Note:
% The \lipsum[#] commands throughout this template generate dummy text
% to fill the template out. These commands should all be removed when
% writing assignment content.
%
%%%%%%%%%%%%%%%%%%%%%%%%%%%%%%%%%%%%%%%%%

%----------------------------------------------------------------------------------------
%	PACKAGES AND OTHER DOCUMENT CONFIGURATIONS
%----------------------------------------------------------------------------------------

\documentclass{article}

\usepackage{fancyhdr} % Required for custom headers
\usepackage{lastpage} % Required to determine the last page for the footer
\usepackage{extramarks} % Required for headers and footers
\usepackage{graphicx} % Required to insert images
\usepackage{lipsum} % Used for inserting dummy 'Lorem ipsum' text into the template
\usepackage{enumerate}
\usepackage{booktabs}
\usepackage{amsmath}
\usepackage{amssymb}

% Margins
\topmargin=-0.45in
\evensidemargin=0in
\oddsidemargin=0in
\textwidth=6.5in
\textheight=9.0in
\headsep=0.25in

\linespread{1.5} % Line spacing

% Set up the header and footer
\pagestyle{fancy}
\lhead{\hmwkAuthorName} % Top left header
\chead{\hmwkClass\ (\hmwkTitle)} % Top center header
%%\rhead{\firstxmark}
\rhead{} % Top right header
\lfoot{\lastxmark} % Bottom left footer
\cfoot{} % Bottom center footer
\rfoot{Page\ \thepage\ of\ \pageref{LastPage}} % Bottom right footer
\renewcommand\headrulewidth{0.4pt} % Size of the header rule
\renewcommand\footrulewidth{0.4pt} % Size of the footer rule

\setlength\parindent{0pt} % Removes all indentation from paragraphs

%----------------------------------------------------------------------------------------
%	DOCUMENT STRUCTURE COMMANDS
%	Skip this unless you know what you're doing
%----------------------------------------------------------------------------------------

% Header and footer for when a page split occurs within a problem environment
\newcommand{\enterProblemHeader}[1]{
\nobreak\extramarks{#1}{#1 continued on next page\ldots}\nobreak
\nobreak\extramarks{#1 (continued)}{#1 continued on next page\ldots}\nobreak
}

% Header and footer for when a page split occurs between problem environments
\newcommand{\exitProblemHeader}[1]{
\nobreak\extramarks{#1 (continued)}{#1 continued on next page\ldots}\nobreak
\nobreak\extramarks{#1}{}\nobreak
}

\setcounter{secnumdepth}{0} % Removes default section numbers
\newcounter{homeworkProblemCounter} % Creates a counter to keep track of the number of problems

\newcommand{\homeworkProblemName}{}
\newenvironment{homeworkProblem}[1][Problem \arabic{homeworkProblemCounter}]{ % Makes a new environment called homeworkProblem which takes 1 argument (custom name) but the default is "Problem #"
\stepcounter{homeworkProblemCounter} % Increase counter for number of problems
\renewcommand{\homeworkProblemName}{#1} % Assign \homeworkProblemName the name of the problem
\section{\homeworkProblemName} % Make a section in the document with the custom problem count
\enterProblemHeader{\homeworkProblemName} % Header and footer within the environment
}{
\exitProblemHeader{\homeworkProblemName} % Header and footer after the environment
}

\newcommand{\problemAnswer}[1]{ % Defines the problem answer command with the content as the only argument
\noindent\framebox[\columnwidth][c]{\begin{minipage}{0.98\columnwidth}#1\end{minipage}} % Makes the box around the problem answer and puts the content inside
}

\newcommand{\homeworkSectionName}{}
\newenvironment{homeworkSection}[1]{ % New environment for sections within homework problems, takes 1 argument - the name of the section
\renewcommand{\homeworkSectionName}{#1} % Assign \homeworkSectionName to the name of the section from the environment argument
\subsection{\homeworkSectionName} % Make a subsection with the custom name of the subsection
\enterProblemHeader{\homeworkProblemName\ [\homeworkSectionName]} % Header and footer within the environment
}{
\enterProblemHeader{\homeworkProblemName} % Header and footer after the environment
}

%----------------------------------------------------------------------------------------
%	PARTIAL DERIVATIVES
%----------------------------------------------------------------------------------------
\newcommand{\pdv}[3][]{
	\frac{\partial^{#1}{#2}}{\partial{{#3}^{#1}}}
}

%----------------------------------------------------------------------------------------
%	EXPECTATION AND VARIANCE OPERATOR
%----------------------------------------------------------------------------------------
 \newcommand{\E}{\mathrm{E}}
 \newcommand{\Var}{\mathrm{Var}}
 \newcommand{\Cov}{\mathrm{Cov}}
 \newcommand{\Corr}{\mathrm{Corr}}

%----------------------------------------------------------------------------------------
%	ITO's LEMMA
%----------------------------------------------------------------------------------------
\newcommand{\itox}[2]{ \pdv{#1}{#2}d#2 + \frac{1}{2}\pdv[2]{#1}{#2}(d#2)^2}
\newcommand{\itotx}[2]{\pdv{#1}{t}dt + \pdv{#1}{#2}d#2 + \frac{1}{2}\pdv[2]{#1}{#2}(d#2)^2}

%----------------------------------------------------------------------------------------
%	NAME AND CLASS SECTION
%----------------------------------------------------------------------------------------

\newcommand{\hmwkTitle}{Problem Set\ \#5} % Assignment title
\newcommand{\hmwkDueDate}{Tuesday,\ March\ 13,\ 2018} % Due date
\newcommand{\hmwkClass}{FIN\ 514} % Course/class
\newcommand{\hmwkClassTime}{9:30am} % Class/lecture time
\newcommand{\hmwkAuthorName}{Wanbae Park} % Your name

%----------------------------------------------------------------------------------------
%	TITLE PAGE
%----------------------------------------------------------------------------------------

\title{
\vspace{2in}
\textmd{\textbf{\hmwkClass:\ \hmwkTitle}}\\
\normalsize\vspace{0.1in}\small{Due\ on\ \hmwkDueDate}\\
\vspace{3in}
}

\author{\textbf{\hmwkAuthorName}}
\date{} % Insert date here if you want it to appear below your name

%----------------------------------------------------------------------------------------

\begin{document}

\maketitle

%----------------------------------------------------------------------------------------
%	TABLE OF CONTENTS
%----------------------------------------------------------------------------------------

%\setcounter{tocdepth}{1} % Uncomment this line if you don't want subsections listed in the ToC

%%\newpage
%%\tableofcontents
\newpage

%----------------------------------------------------------------------------------------
%	PROBLEM 1
%----------------------------------------------------------------------------------------

% To have just one problem per page, simply put a \clearpage after each problem

\begin{homeworkProblem}
	Option price must satisfy the following pde. It is analogous that diffusion term has changed from $\sigma S(t)$ to $\sigma (S(t))^\gamma$.
	\begin{equation*}
		\pdv{V}{t} + \frac{1}{2} \sigma^2 S^{2\gamma} \pdv[2]{V}{S} + rS \pdv{V}{S} - rV = 0
	\end{equation*}
	It can be derived by following procedures.	\\
	By Ito's lemma, $dV = \pdv{V}{t}dt + \pdv{V}{S}dS + \frac{1}{2}\sigma^2 S^{2\gamma} \pdv[2]{V}{S}dt$. Set $\Pi = V - \Delta S - \beta B$,
	then by self-financing, $d\Pi = (\pdv{V}{t} + \frac{1}{2}\sigma^2 S^{2\gamma} \pdv[2]{V}{S})dt + \pdv{V}{S}dS - \Delta dS - r\beta B dt$.
	Choose $\Delta$ such that $(\pdv{V}{S} - \Delta)dS = 0$, therefore $\Delta = \pdv{V}{S}$.
	Since $\beta B = V - \Delta S$, and by no arbirage argument, $0 = d\Pi = (\pdv{V}{t} + \frac{1}{2}\sigma^2 S^{2\gamma} \pdv[2]{V}{S} + rS\pdv{V}{S} - rV)$.
\end{homeworkProblem}

%----------------------------------------------------------------------------------------
%	PROBLEM 2
%----------------------------------------------------------------------------------------
\begin{homeworkProblem}
\begin{enumerate}[(a)]
	\item	%% Problem (a)
		By Black-Scholes, all derivatives with underlying asset $S$ must follow the following pde.
		\begin{equation*}
			\pdv{V}{t} + \frac{1}{2} \sigma^2 S^2 \pdv[2]{V}{S} + rS \pdv{V}{S} - rV = 0
		\end{equation*}
		Terminal boundary condition of this contract is $V(S, T) = \ln(S(T) / S(0))$.
	\item	%% Problem (b)
		By Ito's lemma,
		\begin{equation*}
		\begin{aligned}
			d\ln S 	&= \frac{1}{S}dS - \frac{1}{2} \frac{1}{S^2}(dS)^2	\\
					&= (\mu - \frac{1}{2}\sigma^2)dt + \sigma dX(t)		\\
			\Rightarrow S(t) &= S(0)\exp((\mu - \frac{1}{2}\sigma^2)t + \sigma X(t))	\\
			\Rightarrow V(S,t) &= \ln(S(t) / S(0)) = (\mu - \frac{1}{2}\sigma^2)t + \sigma X(t)
		\end{aligned}
		\end{equation*}
\end{enumerate}
\end{homeworkProblem}

%----------------------------------------------------------------------------------------
%	PROBLEM 3
%----------------------------------------------------------------------------------------
\begin{homeworkProblem}
\begin{enumerate}[(a)]
	\item	%% Problem (a)
		By Ito's product rule,
		\begin{equation*}
		\begin{aligned}
			dS_D	&= d(eS)	\\
					&= edS + Sde + dSde	\\
					&= e(\mu Sdt + \sigma S dX_2) + S(\mu_e edt + \sigma_e e dX_1) + (\mu Sdt + \sigma S dX_2)(\mu_e edt + \sigma_e e dX_1)	\\
					&= (\mu eS + \mu_e eS + \sigma \sigma_e \rho eS)dt + \sigma_e eSdX_1 + \sigma eS dX_2	\\
					&= (\mu + \mu_e + \sigma \sigma_e \rho)S_D dt + \sigma_e S_D dX_1 + \sigma S_D dX_2
		\end{aligned}
		\end{equation*}
	\item	%% Problem (b)
		By Ito's product rule,
		\begin{equation*}
		\begin{aligned}
			dB_{KD}		&= d(eB_K)	\\
						&= edB_K + B_Kde + dB_Kde	\\
						&= e(r_K B_K dt) + B_K(\mu_e edt + \sigma_e e dX_1) + (r_K B_K dt)(\mu_e edt + \sigma_e e dX_1)	\\
						&= (r_K eB_K + \mu_e eB_K)dt + \sigma_e eB_K dX_1	\\
						&= (r_K + \mu_e)B_{KD}dt + \sigma_e B_{KD} dX_1
		\end{aligned}
		\end{equation*}
\end{enumerate}
\end{homeworkProblem}

%----------------------------------------------------------------------------------------
%	PROBLEM 4
%----------------------------------------------------------------------------------------
\begin{homeworkProblem}
\begin{enumerate}[(a)]
	\item	%% problem (a)
	By Ito's product rule, $dY_t = d(\frac{S_{1t}}{S_{2t}})
	= S_{2t}d(\frac{1}{S_{1t}}) + \frac{1}{S_{1t}}dS_{2t} + d(\frac{1}{S_{1t}})dS_{2t}$.
	And by Ito's lemma,
	\begin{equation*}
	\begin{aligned}
		d(\frac{1}{S_{1t}})	&= -\frac{1}{S_{1t}^2}dS_{1t} +
								\frac{1}{2} \times 2 \times \frac{1}{S_{1t}^3}(dS_{1t})^2	\\
							&= -\frac{1}{S_{1t}^2}(\mu_1 S_{1t}dt + \sigma_1 S_{1t} dX_{1t})
								+ \frac{1}{S_{1t}^3}\sigma_1^2 S_{1t}^2 dt	\\
							&= (-\mu_1 + \sigma_1^2) \frac{1}{S_{1t}}dt - \sigma_1 \frac{1}{S_{1t}}dX_{1t}
	\end{aligned}
	\end{equation*}
	Therefore, the following equation must hold.
	\begin{equation*}
		\begin{aligned}
			dY_t	&= S_{2t}(-\mu_1 + \sigma_1^2) \frac{1}{S_{1t}}dt - \sigma_1 \frac{1}{S_{1t}}dX_{1t})
						+ \frac{1}{S_{1t}}(\mu_2 S_{2t}dt + \sigma_2 S_{2t}dX_{2t})	\\
						~~ &+ (-\mu_1 + \sigma_1^2) \frac{1}{S_{1t}}dt - \sigma_1 \frac{1}{S_{1t}}dX_{1t})
						(\frac{1}{S_{1t}}(\mu_2 S_{2t}dt + \sigma_2 S_{2t}dX_{2t}))	\\
					&= (-\mu_1 + \sigma_1^2)Y_tdt - \sigma_1 Y_t dX_{1t}
						+ \mu_2 Y_t dt + \sigma_2 Y_t dX_{1t}
						- \sigma_1\sigma_2\rho Y_t dt ~~ (\text{Since~} dX_{1t}dX_{2t} = \rho dt).	\\
					&= (-\mu_1 + \mu_2 + \sigma_1^2 - \sigma_1\sigma_2\rho)Y_tdt
						- \sigma_1Y_tdX_{1t} + \sigma_2Y_tdX_{2t}
		\end{aligned}
	\end{equation*}
	If we choose $X_{3t} = \frac{-\sigma_1 X_{1t} + \sigma_2 X_{2t}}{\sqrt{\sigma_1^2 + \sigma_2^2 - 2\sigma_1\sigma_2\rho}}$,
	then the equation above is represented as follows.
	\begin{equation*}
		\begin{aligned}
			dY_t &= \mu_Y Y_t dt + \sigma_Y Y_t dX_{3t}	\\
			\text{where} ~~ \mu_Y &= -\mu_1 + \mu_2 + \sigma_1^2 - \sigma_1\sigma_2\rho,
			\sigma_Y = \sqrt{\sigma_1^2 + \sigma_2^2 - 2\sigma_1\sigma_2\rho}
		\end{aligned}
	\end{equation*}
	\item	%% problem (b)
	By Ito's product rule,
	\begin{equation*}
		\begin{aligned}
			dZ_t 	&= d(Y_tB_{Et})	\\
					&= B_{Et}dY_t + Y_tdB_{Et} + dY_tdB_{Et}	\\
					&= B_{Et}(\mu_Y Y_tdt + \sigma_Y Y_t dX_{3t})
						+ Y_t(r_E B_{Et}dt)
						+ (\mu_Y Y_tdt + \sigma_Y Y_t dX_{3t})(Y_tr_E B_{Et}dt)	\\
					&= (\mu_Y + r_E)Y_tB_{Et}dt + \sigma_Y Y_tB_{Et}dX_{3t}	\\
					&= \mu_Z Z_t dt + \sigma_Y Z_t dX_{3t}	\\
					\text{where} ~~ \mu_Z &= \mu_Y + r_E.
		\end{aligned}
	\end{equation*}
	\item	%% problem (c)
	$U(Y, T) = \frac{V(S_{1T}, S_{2T}, T)}{S_{1T}}
				= \frac{1000 \times \max(S_{2T} - S_{1T}, 0)}{S_{1T}}
				= 1000 \times \max(Y_T - 1, 0)$.
	Payoff of exchange option denominated in GBP is same as payoff of an European option whose underlying asset is
	price of Euros in terms of GBP and strike price is 1.
	\item	%% problem (d)
	Let $\Pi_t = \frac{\pi_t}{S_{1t}}$, then $\Pi_t = -U_t + \Delta_tZ_t + \beta_t B_{Pt}$.
	Therefore, the following equation has to follow.
	\begin{equation*}
		\begin{aligned}
			d\Pi_t 	&= -dU_t + \Delta_t dZ_t + \beta_tdB_{Pt} ~~ \text{by self-financing.}	\\
					&= -(\pdv{U_t}{t} + \mu_Y Y_t \pdv{U_t}{Y_t} + \frac{1}{2}\sigma_Y^2 Y^2_t \pdv[2]{U_t}{Y_t})dt
						+\Delta_t \mu_Z Z_tdt + \beta_t r_P B_{Pt}dt
						-\sigma_Y Y_t \pdv{U_t}{Y_t}dX_{3t} + \Delta_t \sigma_Y Z_t dX_{3t}
		\end{aligned}
	\end{equation*}
	Choose $\Delta_t$ so that $-\sigma_Y Y_t \pdv{U_t}{Y_t} + \Delta_t \sigma_Y Z_t = 0$, then
	$\Delta_t = \frac{Y_t}{Z_t}\pdv{U_t}{Y_t} = B_{Et} \pdv{U_t}{Y_t}$, and therefore,
	\begin{equation*}
		\begin{aligned}
			-(\pdv{U_t}{t} + \mu_Y Y_t \pdv{U_t}{Y_t} + \frac{1}{2}\sigma_Y^2 Y_t^2 \pdv[2]{U_t}{Y_t})dt
				+ \frac{Y_t}{Z_t}\pdv{U_t}{Z_t}\mu_Z Z_tdt
				+ r_P(U_t - \frac{Y_t}{Z_t}\pdv{U_t}{Y_t} Z_t)dt = 0	\\
			\Rightarrow \pdv{U_t}{t} + \mu_Y Y_t \pdv{U_t}{Y_t} + \frac{1}{2}\sigma_Y^2 Y_t^2 \pdv[2]{U_t}{Y_t}
				- \mu_Z Y_t \pdv{U_t}{Y_t} + r_P Y_t \pdv{U_t}{Y_t} - r_PU_t = 0	\\
			\Rightarrow \pdv{U_t}{t} + (\mu_Y - \mu_Z + r_P)Y_t\pdv{U_t}{Y_t}
				+ \frac{1}{2}\sigma_Y^2 Y_t^2 \pdv[2]{U_t}{Y_t} - r_P U_t = 0
		\end{aligned}
	\end{equation*}
	Since $\mu_Y - \mu_Z = -r_E$, the following pde holds.
	\begin{equation*}
		\begin{aligned}
			\pdv{U_t}{t} + (r_P - r_E)Y_t\pdv{U_t}{Y_t}
				+ \frac{1}{2}\sigma_Y^2 Y_t^2 \pdv[2]{U_t}{Y_t} - r_P U_t = 0
		\end{aligned}
	\end{equation*}
	\item	%% problem (e)
	Since $U(Y, T) = 1000 \times \max(Y_T - 1, 0)$, by analogy of European call option,
	$U(Y, 0)$ is evaluated as:
	\begin{equation*}
		\begin{aligned}
			U(Y, 0)  &= e^{-r_PT}(Y_te^{(r_P - r_E)T} N(d_1) - N(d_2))	\\
			\text{where} ~~ d_1 &= \frac{\ln Y_t + (r_P - r_E + \frac{1}{2}\sigma_Y^2)T}{\sigma_Y \sqrt{T}}	\\
			d_2 &= d_1 - \sigma_Y \sqrt{T}
		\end{aligned}
	\end{equation*}
	Therefore, $V(S_1, S_2, 0) = S_{10} \times U(Y, 0)$ is evaluated as:
	\begin{equation*}
		\begin{aligned}
			V(S_1, S_2, 0) 	&= S_{10} e^{-r_PT}(Y_te^{(r_P - r_E)T} N(d_1) - N(d_2))	\\
							&= e^{-r_PT}(S_{20}e^{(r_P - r_E)T} N(d_1) - S_{10}N(d_2))
		\end{aligned}
	\end{equation*}
\end{enumerate}
\end{homeworkProblem}

%----------------------------------------------------------------------------------------
%	PROBLEM 5
%----------------------------------------------------------------------------------------
\begin{homeworkProblem}
	\begin{enumerate}[(a)]
		\item	%% problem (a)
		By Ito's product rule,
		\begin{equation*}
			\begin{aligned}
				dY 	&= d(S_1S_2)	\\
					&= S_1dS_2 + S_2dS_1 + dS_1dS_2	\\
					&= S_1(\mu_2 S_2 dt + \sigma_2 S_2 dX_2) + S_2(\mu_1 S_1 dt + \sigma_1 S_1 dX_1)
						+ (\mu_1 S_1 dt + \sigma_1 S_1 dX_1)(\mu_2 S_2 dt + \sigma_2 S_2 dX_2)	\\
					&= (\mu_1 + \mu_2 + \sigma_1 \sigma_2 \rho)Y_tdt
						+ \sigma_1 Y dX_1 + \sigma_2 Y dX_2	\\
					&= \mu_Y Ydt + \sigma_Y Y dX_3	\\
				\text{where} ~~ \mu_Y &= \mu_1 + \mu_2 + \sigma_1 \sigma_2 \rho,
					\sigma_Y = \sqrt{\sigma_1^2 + \sigma_2^2 + 2\sigma_1 \sigma_2 \rho},
					X_3 = \frac{\sigma_1 X_1 + \sigma_2 X_2}{\sqrt{\sigma_1^2 + \sigma_2^2 + 2\sigma_1 \sigma_2 \rho}}
			\end{aligned}
		\end{equation*}
		\item	%% problem (b)
		Payoff can be represented as $\max((S_1S_2)^{\frac{1}{2}} - K, 0) = \max(Y^{\frac{1}{2}} - K, 0)$.
		\item	%% problem (c)
		By Ito's lemma,
		\begin{equation*}
			\begin{aligned}
				dZ 	&= \pdv{Z}{Y}dY + \frac{1}{2}\pdv[2]{Z}{Y}(dY)^2	\\
					&= \frac{1}{2}Y^{-\frac{1}{2}} (\mu_Y Y dt + \sigma_Y Y dX_3)
						+ \frac{1}{2} (-\frac{1}{4}) Y^{-\frac{3}{2}} \sigma_Y^2 Y^2 dt	\\
					&= \frac{1}{2}(\mu_Y - \frac{1}{4} \sigma_Y^2)Y^{\frac{1}{2}}dt
						+ \frac{1}{2} \sigma_Y Y^{\frac{1}{2}}dX_3
			\end{aligned}
		\end{equation*}
		\item	%% problem (d)
		Since we assume self-financing portfolio,
		\begin{equation*}
			\begin{aligned}
				d\pi	&= -dV + \Delta dZ + \beta dB	\\
						&= -(\pdv{V}{t}dt + \pdv{V}{Z}dZ + \frac{1}{2}\sigma_Y^2 Z^2 dt)
							+ \Delta dZ + \beta rBdt
			\end{aligned}
		\end{equation*}
		Choose $\Delta = \pdv{V}{Z}$, then
		\begin{equation*}
			\begin{aligned}
				d\pi = -(\pdv{Z}{t} + \frac{1}{2}\sigma_Y^2 Z^2)dt
					+ r(V - \pdv{V}{Z})dt = 0	\\
				\Rightarrow \pdv{V}{t} + \frac{1}{2}\sigma_Y^2 Z^2 + rZ \pdv{V}{Z} - rV = 0
			\end{aligned}
		\end{equation*}
		Therefore, $V(Z, t)$ follows Black-Scholes pde in terms of Z.
		\item	%% problem (e)
		From the general Black-Scholes pde: $\pdv{V}{t} + \frac{1}{2}a^2X^2 \pdv{V}{X} + bX \pdv{V}{X} - cV = 0$,
		the risk-neutral expected return of underlying asset is $b$,
		the coefficient of first partial derivatives.
		Therefore, risk-neutral expeced return of geometic average of two stocks is $r$, the risk-free rate.
		\item	%% problem (f)
		By analogy of evaluating European call option from Black-Scholes pde,
		value of geometric average call option is evaluated as follows.
		\begin{equation*}
			\begin{aligned}
				V(S_1, S_2, t) 	&= e^{-r(T - t)}(Ze^{r(T - t)}N(d_1) - KN(d_2))	\\
								&= e^{-r(T - t)}((S_1S_2)^{\frac{1}{2}})e^{r(T - t)}N(d_1) - KN(d_2))	\\
				\text{where} ~~ d_1 &= \frac{\ln((S_1S_2)^\frac{1}{2} / K) + (r + \frac{1}{2}\sigma_Y^2)(T - t)}{\sigma_Y \sqrt{T - t}}	\\
								d_2 &= d_1 - \sigma_Y \sqrt{T - t}
			\end{aligned}
		\end{equation*}
		\item	%% problem (g)
		If correlation between two stocks increases, option value also increases.
		It is because if correlation gets higher, volatility of geometric average also gets higher,
		and option value increases as volatility increases. If there is negative correlation between stocks,
		although one of them increases, the other decreases, therefore geometric average does not changes much.
		Mathematically, since $\sigma_Y = \sqrt{\sigma_1^2 + \sigma_2^2 + 2\sigma_1 \sigma_2 \rho}$,
		if correlation $\rho$ increases, $\sigma_Y$ increases, hence option value increases.
	\end{enumerate}
\end{homeworkProblem}
\end{document}
