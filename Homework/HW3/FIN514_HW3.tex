%%%%%%%%%%%%%%%%%%%%%%%%%%%%%%%%%%%%%%%%%
% Structured General Purpose Assignment
% LaTeX Template
%
% This template has been downloaded from:
% http://www.latextemplates.com
%
% Original author:
% Ted Pavlic (http://www.tedpavlic.com)
%
% Note:
% The \lipsum[#] commands throughout this template generate dummy text
% to fill the template out. These commands should all be removed when 
% writing assignment content.
%
%%%%%%%%%%%%%%%%%%%%%%%%%%%%%%%%%%%%%%%%%

%----------------------------------------------------------------------------------------
%	PACKAGES AND OTHER DOCUMENT CONFIGURATIONS
%----------------------------------------------------------------------------------------

\documentclass{article}

\usepackage{fancyhdr} % Required for custom headers
\usepackage{lastpage} % Required to determine the last page for the footer
\usepackage{extramarks} % Required for headers and footers
\usepackage{graphicx} % Required to insert images
\usepackage{lipsum} % Used for inserting dummy 'Lorem ipsum' text into the template
\usepackage{enumerate}
\usepackage{booktabs}
\usepackage{amsmath}

% Margins
\topmargin=-0.45in
\evensidemargin=0in
\oddsidemargin=0in
\textwidth=6.5in
\textheight=9.0in
\headsep=0.25in 

\linespread{1.5} % Line spacing

% Set up the header and footer
\pagestyle{fancy}
\lhead{\hmwkAuthorName} % Top left header
\chead{\hmwkClass\ (\hmwkTitle)} % Top center header
%%\rhead{\firstxmark} 
\rhead{} % Top right header
\lfoot{\lastxmark} % Bottom left footer
\cfoot{} % Bottom center footer
\rfoot{Page\ \thepage\ of\ \pageref{LastPage}} % Bottom right footer
\renewcommand\headrulewidth{0.4pt} % Size of the header rule
\renewcommand\footrulewidth{0.4pt} % Size of the footer rule

\setlength\parindent{0pt} % Removes all indentation from paragraphs

%----------------------------------------------------------------------------------------
%	DOCUMENT STRUCTURE COMMANDS
%	Skip this unless you know what you're doing
%----------------------------------------------------------------------------------------

% Header and footer for when a page split occurs within a problem environment
\newcommand{\enterProblemHeader}[1]{
\nobreak\extramarks{#1}{#1 continued on next page\ldots}\nobreak
\nobreak\extramarks{#1 (continued)}{#1 continued on next page\ldots}\nobreak
}

% Header and footer for when a page split occurs between problem environments
\newcommand{\exitProblemHeader}[1]{
\nobreak\extramarks{#1 (continued)}{#1 continued on next page\ldots}\nobreak
\nobreak\extramarks{#1}{}\nobreak
}

\setcounter{secnumdepth}{0} % Removes default section numbers
\newcounter{homeworkProblemCounter} % Creates a counter to keep track of the number of problems

\newcommand{\homeworkProblemName}{}
\newenvironment{homeworkProblem}[1][Problem \arabic{homeworkProblemCounter}]{ % Makes a new environment called homeworkProblem which takes 1 argument (custom name) but the default is "Problem #"
\stepcounter{homeworkProblemCounter} % Increase counter for number of problems
\renewcommand{\homeworkProblemName}{#1} % Assign \homeworkProblemName the name of the problem
\section{\homeworkProblemName} % Make a section in the document with the custom problem count
\enterProblemHeader{\homeworkProblemName} % Header and footer within the environment
}{
\exitProblemHeader{\homeworkProblemName} % Header and footer after the environment
}

\newcommand{\problemAnswer}[1]{ % Defines the problem answer command with the content as the only argument
\noindent\framebox[\columnwidth][c]{\begin{minipage}{0.98\columnwidth}#1\end{minipage}} % Makes the box around the problem answer and puts the content inside
}

\newcommand{\homeworkSectionName}{}
\newenvironment{homeworkSection}[1]{ % New environment for sections within homework problems, takes 1 argument - the name of the section
\renewcommand{\homeworkSectionName}{#1} % Assign \homeworkSectionName to the name of the section from the environment argument
\subsection{\homeworkSectionName} % Make a subsection with the custom name of the subsection
\enterProblemHeader{\homeworkProblemName\ [\homeworkSectionName]} % Header and footer within the environment
}{
\enterProblemHeader{\homeworkProblemName} % Header and footer after the environment
}
   
%----------------------------------------------------------------------------------------
%	NAME AND CLASS SECTION
%----------------------------------------------------------------------------------------

\newcommand{\hmwkTitle}{Problem Set\ \#3} % Assignment title
\newcommand{\hmwkDueDate}{Sunday,\ February\ 18,\ 2018} % Due date
\newcommand{\hmwkClass}{FIN\ 514} % Course/class
\newcommand{\hmwkClassTime}{9:30am} % Class/lecture time
\newcommand{\hmwkAuthorName}{Wanbae Park} % Your name

%----------------------------------------------------------------------------------------
%	TITLE PAGE
%----------------------------------------------------------------------------------------

\title{
\vspace{2in}
\textmd{\textbf{\hmwkClass:\ \hmwkTitle}}\\
\normalsize\vspace{0.1in}\small{Due\ on\ \hmwkDueDate}\\
\vspace{3in}
}

\author{\textbf{\hmwkAuthorName}}
\date{} % Insert date here if you want it to appear below your name

%----------------------------------------------------------------------------------------

\begin{document}

\maketitle

%----------------------------------------------------------------------------------------
%	TABLE OF CONTENTS
%----------------------------------------------------------------------------------------

%\setcounter{tocdepth}{1} % Uncomment this line if you don't want subsections listed in the ToC

%%\newpage
%%\tableofcontents
\newpage

%----------------------------------------------------------------------------------------
%	PROBLEM 1
%----------------------------------------------------------------------------------------

% To have just one problem per page, simply put a \clearpage after each problem

\begin{homeworkProblem}
	 %% Sub-Problem a.
	Using the following Black-Scholes formula, the option price was calculated as 10.2479.
	%% Black-Scholes Formula
%----------------------------------------------------------------------------------------
	\begin{equation*}
	\begin{aligned}
		\text{\textit{Put option price}} &= Ke^{-r(T - t)}N(-d_2) - Se^{-\delta (T - t)}N(-d_1)	\\
		d_1 &= \frac{log(\frac{S}{K}) + (r - \delta + \frac{1}{2} \sigma^2)(T - t)}{\sigma \sqrt{T - t}}	\\
		d_2 &= d_1 - \sigma \sqrt{T - t}
	\end{aligned}
	\end{equation*}
%----------------------------------------------------------------------------------------
	Then, using Cox, Ross and Rubinstein(CRR), Rendleman and Bartter(RB), Leisen and Reimer(LR) method each, put option value was calculated from $N = 50$ to $N = 1000$. Figure \ref{fig:prob1-error} shows the error of each method.
	%% Error Figure
	\begin{figure}[ht]
		\centering
		\includegraphics[scale = 0.5]{Q1/error.png}
		\caption{Error of each methods}
		\label{fig:prob1-error}
	\end{figure}
	As shown in figure, except LR method, there seems to exist some problems. In CRR method, it looks like that error is converging to zero, but it is not monotonic. It means it does not guarantee that applying more steps makes more accurate values. Regarding RB method, it seems better than CRR, but error is increasing from some points(about $N = 300$). The reason for this phenomenon is that option payoff is not linear shape. Since LR method solves this problem when $N$ is odd, the shape of error in LR method seems monotonically decreasing to zero as $N$ goes to some large value. The reason why monotonicity is important is that we can extrapolate values from two binomial trees to get more accurate values if monotonic error is guaranteed. Figure \ref{fig:prob1-error_extra} shows the error after extrapolation($M = 2N$ is used when using CRR and RB method, $M = 2N - 1$ is used for extrapolation.).
	%% Extrapolation
	\begin{figure}[ht]
		\centering
		\includegraphics[scale = 0.5]{Q1/errorextra.png}
		\caption{Error of each methods after extrapolation}
		\label{fig:prob1-error_extra}
	\end{figure}
	Since it is well-known that LR method has $O(1/n^2)$ errors, the extrapolation procedure has changed from original one to followings.
	\begin{equation*}
		V_{EXACT} \approx \frac{M^2 V_M - N^2 V_N}{M^2 - N^2} ~~ \text{where $M$ and $N$ are odd numbers.}
	\end{equation*}
	%% Accuracy of extrapolation: CRR, RB(X), LR(O) because of monotonicity
	As shown in Figure \ref{fig:prob1-error_extra}, the error of CRR and RB methods seems sawtoothing, but error of LR method is converging to zero monotonically. Furthermore, the accuracy of value is even worse at some points for CRR and RB method when extrapolation is applied. Before using extrapolation technique, the maximum error of CRR and RB method is about 0.05, but there are some points where error is about 0.1 after extrapolation. However, in LR method, the error after extrapolation is always smaller than before. That is why monotonicity is important when using extrapolation to get more accurate value.
\end{homeworkProblem}

%----------------------------------------------------------------------------------------
%	PROBLEM 2
%----------------------------------------------------------------------------------------
\begin{homeworkProblem}

\end{homeworkProblem}
\end{document}