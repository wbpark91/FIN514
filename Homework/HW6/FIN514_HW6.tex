%%%%%%%%%%%%%%%%%%%%%%%%%%%%%%%%%%%%%%%%%
% Structured General Purpose Assignment
% LaTeX Template
%
% This template has been downloaded from:
% http://www.latextemplates.com
%
% Original author:
% Ted Pavlic (http://www.tedpavlic.com)
%
% Note:
% The \lipsum[#] commands throughout this template generate dummy text
% to fill the template out. These commands should all be removed when
% writing assignment content.
%
%%%%%%%%%%%%%%%%%%%%%%%%%%%%%%%%%%%%%%%%%

%----------------------------------------------------------------------------------------
%	PACKAGES AND OTHER DOCUMENT CONFIGURATIONS
%----------------------------------------------------------------------------------------

\documentclass{article}

\usepackage{fancyhdr} % Required for custom headers
\usepackage{lastpage} % Required to determine the last page for the footer
\usepackage{extramarks} % Required for headers and footers
\usepackage{graphicx} % Required to insert images
\usepackage{lipsum} % Used for inserting dummy 'Lorem ipsum' text into the template
\usepackage{enumerate}
\usepackage{booktabs}
\usepackage{amsmath}
\usepackage{amssymb}

% Margins
\topmargin=-0.45in
\evensidemargin=0in
\oddsidemargin=0in
\textwidth=6.5in
\textheight=9.0in
\headsep=0.25in

\linespread{1.5} % Line spacing

% Set up the header and footer
\pagestyle{fancy}
\lhead{\hmwkAuthorName} % Top left header
\chead{\hmwkClass\ (\hmwkTitle)} % Top center header
%%\rhead{\firstxmark}
\rhead{} % Top right header
\lfoot{\lastxmark} % Bottom left footer
\cfoot{} % Bottom center footer
\rfoot{Page\ \thepage\ of\ \pageref{LastPage}} % Bottom right footer
\renewcommand\headrulewidth{0.4pt} % Size of the header rule
\renewcommand\footrulewidth{0.4pt} % Size of the footer rule

\setlength\parindent{0pt} % Removes all indentation from paragraphs

%----------------------------------------------------------------------------------------
%	DOCUMENT STRUCTURE COMMANDS
%	Skip this unless you know what you're doing
%----------------------------------------------------------------------------------------

% Header and footer for when a page split occurs within a problem environment
\newcommand{\enterProblemHeader}[1]{
\nobreak\extramarks{#1}{#1 continued on next page\ldots}\nobreak
\nobreak\extramarks{#1 (continued)}{#1 continued on next page\ldots}\nobreak
}

% Header and footer for when a page split occurs between problem environments
\newcommand{\exitProblemHeader}[1]{
\nobreak\extramarks{#1 (continued)}{#1 continued on next page\ldots}\nobreak
\nobreak\extramarks{#1}{}\nobreak
}

\setcounter{secnumdepth}{0} % Removes default section numbers
\newcounter{homeworkProblemCounter} % Creates a counter to keep track of the number of problems

\newcommand{\homeworkProblemName}{}
\newenvironment{homeworkProblem}[1][Problem \arabic{homeworkProblemCounter}]{ % Makes a new environment called homeworkProblem which takes 1 argument (custom name) but the default is "Problem #"
\stepcounter{homeworkProblemCounter} % Increase counter for number of problems
\renewcommand{\homeworkProblemName}{#1} % Assign \homeworkProblemName the name of the problem
\section{\homeworkProblemName} % Make a section in the document with the custom problem count
\enterProblemHeader{\homeworkProblemName} % Header and footer within the environment
}{
\exitProblemHeader{\homeworkProblemName} % Header and footer after the environment
}

\newcommand{\problemAnswer}[1]{ % Defines the problem answer command with the content as the only argument
\noindent\framebox[\columnwidth][c]{\begin{minipage}{0.98\columnwidth}#1\end{minipage}} % Makes the box around the problem answer and puts the content inside
}

\newcommand{\homeworkSectionName}{}
\newenvironment{homeworkSection}[1]{ % New environment for sections within homework problems, takes 1 argument - the name of the section
\renewcommand{\homeworkSectionName}{#1} % Assign \homeworkSectionName to the name of the section from the environment argument
\subsection{\homeworkSectionName} % Make a subsection with the custom name of the subsection
\enterProblemHeader{\homeworkProblemName\ [\homeworkSectionName]} % Header and footer within the environment
}{
\enterProblemHeader{\homeworkProblemName} % Header and footer after the environment
}

%----------------------------------------------------------------------------------------
%	PARTIAL DERIVATIVES
%----------------------------------------------------------------------------------------
\newcommand{\pdv}[3][]{
	\frac{\partial^{#1}{#2}}{\partial{{#3}^{#1}}}
}

%----------------------------------------------------------------------------------------
%	EXPECTATION AND VARIANCE OPERATOR
%----------------------------------------------------------------------------------------
 \newcommand{\E}{\mathrm{E}}
 \newcommand{\Var}{\mathrm{Var}}
 \newcommand{\Cov}{\mathrm{Cov}}
 \newcommand{\Corr}{\mathrm{Corr}}

%----------------------------------------------------------------------------------------
%	ITO's LEMMA
%----------------------------------------------------------------------------------------
\newcommand{\itox}[2]{ \pdv{#1}{#2}d#2 + \frac{1}{2}\pdv[2]{#1}{#2}(d#2)^2}
\newcommand{\itotx}[2]{\pdv{#1}{t}dt + \pdv{#1}{#2}d#2 + \frac{1}{2}\pdv[2]{#1}{#2}(d#2)^2}

%----------------------------------------------------------------------------------------
%	BROWNIAN MOTION
%----------------------------------------------------------------------------------------
\newcommand{\til}[1]{\widetilde{#1}}

%----------------------------------------------------------------------------------------
%	NAME AND CLASS SECTION
%----------------------------------------------------------------------------------------

\newcommand{\hmwkTitle}{Problem Set\ \#6} % Assignment title
\newcommand{\hmwkDueDate}{Wednesday,\ May\ 2,\ 2018} % Due date
\newcommand{\hmwkClass}{FIN\ 514} % Course/class
\newcommand{\hmwkClassTime}{9:30am} % Class/lecture time
\newcommand{\hmwkAuthorName}{Wanbae Park} % Your name

%----------------------------------------------------------------------------------------
%	TITLE PAGE
%----------------------------------------------------------------------------------------

\title{
\vspace{2in}
\textmd{\textbf{\hmwkClass:\ \hmwkTitle}}\\
\normalsize\vspace{0.1in}\small{Due\ on\ \hmwkDueDate}\\
\vspace{3in}
}

\author{\textbf{\hmwkAuthorName}}
\date{} % Insert date here if you want it to appear below your name

%----------------------------------------------------------------------------------------

\begin{document}

\maketitle

%----------------------------------------------------------------------------------------
%	TABLE OF CONTENTS
%----------------------------------------------------------------------------------------

%\setcounter{tocdepth}{1} % Uncomment this line if you don't want subsections listed in the ToC

%%\newpage
%%\tableofcontents
\newpage

%----------------------------------------------------------------------------------------
%	PROBLEM 1
%----------------------------------------------------------------------------------------

% To have just one problem per page, simply put a \clearpage after each problem

\begin{homeworkProblem}
	\begin{enumerate}[(a)]
		\item	%% (a)
		By Ito's product rule, $dY(t)$ satisfies the following equation.
		%% EQUATION: DYNAMICS OF Y(t)
		\begin{equation*}
			\begin{aligned}
				dY(t)	&= B_P(t)dS(t) + S(t)dB_P(t) + dB_P(t)dS(t)	\\
						&= B_P(t)[\mu S(t)dt + \sigma S(t)dX(t)] + r_PS(t)B_P(t)dt	\\
						&= (\mu + r_P)Y(t)dt + \sigma Y(t)dX(t)
			\end{aligned}
		\end{equation*}
		In order to find martingale measure with respect to $B(t)$ as a numeraire,
		dynamics of $Y(t) / B(t)$ is derived as follows.
		%% EQUATION: DYNAMICS OF Y(t) / B(t)
		\begin{equation*}
			\begin{aligned}
				d \left( \frac{Y(t)}{B(t)} \right)
					&= Y(t) d\left( \frac{1}{B(t)} \right)
						+ \frac{1}{B(t)}dY(t) + dY(t) d \left( \frac{1}{B(t)} \right)	\\
				d \left( \frac{1}{B(t)} \right)
					&= -\frac{1}{B^2(t)}dB(t)	\\
					&= -\frac{1}{B^2(t)}rB(t)dt = -r \frac{1}{B(t)}dt	\\
				\Rightarrow d \left( \frac{Y(t)}{B(t)} \right)
					&= Y(t) \left( -r \frac{1}{B(t)}dt \right)
						+ \frac{1}{B(t)} [(\mu + r_P)Y(t)dt + \sigma Y(t)dX(t)]	\\
					&= (\mu + r_P - r) \frac{Y(t)}{B(t)}dt + \sigma \frac{Y(t)}{B(t)}dX(t)
			\end{aligned}
		\end{equation*}
		By Girsanov's theorem, there exists a probability measure such that
		$\til{X}(t) = X(t) + \int_0^t \frac{\mu + r_P - r}{\sigma} ds$ is a
		brownian motion under the measure. Therefore, by plugging
		$dX(t) = d\til{X}(t) - \frac{\mu + r_P - r}{\sigma}dt$ into the equation
		above, then $d\left( \frac{Y(t)}{B(t)} \right)$ becomes $\sigma \frac{Y(t)}{B(t)}d\til{X}(t)$,
		hence becomes martingale because there is no drift.
		Therefore, from the perspective of U.S dollar investor,
		under risk-neutral measure,
		$dX(t) = d\til{X}(t) - \frac{\mu + r_P - r}{\sigma}dt$.
		By plugging it into dynamics of $Y(t)$, we can find dynamics of the U.S
		price of a GBP bond under risk-neutral measure as follows.
		%% EQUATION: DYNAMICS of Y(t) UNDER Q
		\begin{equation*}
			\begin{aligned}
				dY(t)	&= (\mu + r_P)Y(t)dt + \sigma Y(t)dX(t)	\\
						&= (\mu + r_P)Y(t)dt
							+ \sigma Y(t)\left[ d\til{X}(t) - \frac{\mu + r_P - r}{\sigma}dt \right]	\\
						&= rY(t)dt + \sigma Y(t) d\til{X}(t)
			\end{aligned}
		\end{equation*}
		And it is consistent with the fact that expected return of every tradable
		asset is risk-free rate under risk-neutral measure.
		\item	%% (b)
		By plugging $dX(t) = d\til{X}(t) - \frac{\mu + r_P - r}{\sigma}dt$
		into dynamics of $S(t)$, we can find dynamics of U.S. dollar price
		of a British pound under risk-neutral probability as follows.
		%% DYNAMICS OF S(t) UNDER Q
		\begin{equation*}
			\begin{aligned}
				dS(t)	&= \mu S(t)dt + \sigma S(t) dX(t)	\\
						&= \mu S(t)dt + \sigma S(t) \left[ d\til{X}(t) - \frac{\mu + r_P - r}{\sigma}dt \right]	\\
						&= (r - r_P)S(t)dt + \sigma S(t) d\til{X}(t)
			\end{aligned}
		\end{equation*}
		\item	%% (c)
		Unlike the assumption of ordinary Black-Scholes-Merton formula,
		since expected return of underlying asset has changed from $r$ to
		$r - r_P$, formula for call option should be changed to following
		equation.
		%% EQUATION: FOREX OPTION FORMULA
		\begin{equation*}
			\begin{aligned}
				&e^{-rT}[S_0 e^{(r - r_P)T}N(d_1) - KN(d_2)]	\\
				&d_1 = \frac{\log(S_0 / K) + (r - r_P + \frac{1}{2} \sigma^2)T}{\sigma \sqrt{T}}	\\
				&d_2 = d_1 - \sigma \sqrt{T}
			\end{aligned}
		\end{equation*}
	\end{enumerate}
\end{homeworkProblem}

%----------------------------------------------------------------------------------------
%	PROBLEM 2
%----------------------------------------------------------------------------------------
\begin{homeworkProblem}
	\begin{enumerate}[(a)]
		\item	%% (a)
		By Ito's product rule, dynamics of $B(t) / S(t)$ is as follows.
		%% EQUATION: DYNAMICS OF B(t) / S(t)
		\begin{equation*}
			\begin{aligned}
				d \left( \frac{B(t)}{S(t)} \right)
					&= d\left( \frac{1}{S(t)} \right)B(t)
						+ \frac{1}{S(t)}dB(t)
						+ d\left(\frac{1}{S(t)} \right)dB(t)	\\
				d \left( \frac{1}{S(t)} \right)
					&= -\frac{1}{S^2(t)}dS(t) + \frac{1}{2} \times 2 \times \frac{1}{S^3(t)}(dS(t))^2	\\
					&= -\frac{1}{S^2(t)}[(\mu - d)S(t)dt + \sigma S(t)dX(t)]
						+ \frac{1}{S^3(t)} \sigma^2 S^2(t) dt	\\
					&= [-(\mu - d) + \sigma^2] \frac{1}{S^2(t)}dt
						- \sigma \frac{1}{S(t)}dX(t)	\\
				\Rightarrow d \left( \frac{B(t)}{S(t)} \right)
					&= \frac{B(t)}{S(t)}[-(\mu - d) + \sigma^2]dt
						- \sigma \frac{B(t)}{S(t)} dX(t)
						+ r \frac{B(t)}{S(t)} dt	\\
					&= [r - (\mu - d) + \sigma^2] \frac{B(t)}{S(t)} dt
						- \sigma \frac{B(t)}{S(t)} dX(t)
			\end{aligned}
		\end{equation*}
		\item	%% (b)
		By Girsanov's theorem, there exists a probability measure such that
		$\til{X}(t) = X(t) - \int_0^t \frac{r - (\mu - d) + \sigma^2}{\sigma} ds$
		is a brownian motion under the measure. Under the measure, process
		$d \left( \frac{B(t)}{S(t)} \right)$ changes as follows.
		%% DYNAMICS OF B(t)/S(t) UNDER Q
		\begin{equation*}
			\begin{aligned}
				d \left( \frac{B(t)}{S(t)} \right)
					&= [r - (\mu - d) + \sigma^2] \frac{B(t)}{S(t)} dt
						- \sigma \frac{B(t)}{S(t)} dX(t)	\\
					&= [r - (\mu - d) + \sigma^2] \frac{B(t)}{S(t)} dt
						- \sigma \frac{B(t)}{S(t)} \left[d\til{X}(t)
						+ \frac{r - (\mu - d) + \sigma^2}{\sigma} dt \right]	\\
					&= - \sigma \frac{B(t)}{S(t)} d\til{X}(t)
			\end{aligned}
		\end{equation*}
		Therefore, under the measure, $\frac{B(t)}{S(t)}$ is a martingale
		since there is no drift term in dynamics.
		Plugging $dX(t) = d\til{X}(t) + \frac{r - (\mu - d) + \sigma^2}{\sigma} dt$
		into the process of $S(t)$, dynamics of $S(t)$ under martingale measure
		with respect to $S(t)$ as a numeraire is as follows.
		%% DYNAMICS OF S(t) UNDER Q
		\begin{equation*}
			\begin{aligned}
				dS(t)	&= (\mu - d)S(t)dt + \sigma S(t)dX(t)	\\
						&= (\mu - d)S(t)dt
							+ \sigma S(t)\left[ d\til{X}(t) + \frac{r - (\mu - d)
							+ \sigma^2}{\sigma} dt \right]	\\
						&= (r + \sigma^2)S(t)dt + \sigma S(t) d\til{X}(t)
			\end{aligned}
		\end{equation*}
		\item	%% (c)
		Let $\eta (t) = \E_t^Q[V(T) / S(T)]$. Since conditional expectation is
		always a martingale under corresponding probability measure, $\eta(t)$
		is a $\mathbf{Q}$ - martingale. Then by martingale representation
		theorem, there exists a unique process $\phi(t)$ such that
		$d\eta(t) = \phi(t) d(B(t) / S(t))$. Let us construct a portfolio
		$\Pi(t)$ such that $\Pi(t) = \psi(t)S(t) + \phi(t)B(t)$, where
		$\psi(t) = \eta(t) - \phi(t)B(t) / S(t)$. Then $\Pi(t) = \eta(t)S(t)$
		for all $t$. If $\Pi(t)$ is self-financing, then since there is no
		intermediate cash flow, and $\Pi(T) = \eta(T)S(T) = V(T)$,
		by no arbitrage principle, value of the option at $t$ must be equal to
		$\Pi(t)$. Therefore, it needs to figure out whether $\Pi(t)$ is
		self-financing or not. In order to check it, dynamics of $\Pi(t)$ is
		derived as follows.
		\begin{equation*}
			\begin{aligned}
				d\Pi(t)	&= S(t)d\eta(t) + \eta(t)dS(t) + dS(t)d\eta(t)	\\
						&= \phi(t)S(t) d\left(\frac{B(t)}{S(t)} \right)
							+ \left(\psi(t) + \phi(t) \frac{B(t)}{S(t)} \right)dS(t)
							+ \phi(t) dS(t) d \left( \frac{B(t)}{S(t)} \right)	\\
						&= \phi(t) \left( S(t) d\left( \frac{B(t)}{S(t)} \right)
						  	+ \frac{B(t)}{S(t)}dS(t)
							+ dS(t) d\left( \frac{B(t)}{S(t)} \right)\right)
							+ \psi(t) dS(t)	\\
						&= \phi(t)d\left( S(t) \frac{B(t)}{S(t)}\right)
							+ \psi(t)dS(t)	\\
						&= \psi(t)dS(t) + \phi(t)dB(t)
			\end{aligned}
		\end{equation*}
		From the equation above, we can find out that $\Pi(t)$ is a
		self-financing strategy. Therefore, option value $V(t)$ must equal to
		$\Pi(t) = \eta(t)S(t) = S(t)\E_t^Q[V(T)/S(T)]$, which is represented
		as follows.
		%% EUROPEAN CALL OPTION VALUE UNDER Q
		\begin{equation*}
			\begin{aligned}
				V(t) &= S(t) \E_t^Q \left[ \frac{V(T)}{S(T)} \right]	\\
					&= S(t) \E_t^Q \left[ \frac{\max(S(T) - K, 0)}{S(T)} \right]	\\
					&= S(t) \E_t^Q \left[ \max \left(1 - \frac{K}{S(T)}, 0 \right) \right]
			\end{aligned}
		\end{equation*}
		Where $K$ is strike price of the option, and $Q$ is a probability
		measure in which $B(t)/S(t)$ is a martingale.
		\item	%% (d)
		In order to evaluate call option value, we need to derive the solution
		of SDE $dS(t) = (r + \sigma^2)S(t)dt + \sigma S(t) d\til{X}(t)$ first.
		The solution is derived as follows.
		%% SOLUTION OF SDE
		\begin{equation*}
			\begin{aligned}
				d\log S(t)	&= \frac{1}{S(t)}dS(t) - \frac{1}{2} \frac{1}{S^2(t)}(dS(t))^2	\\
							&= \left( r + \frac{1}{2}\sigma^2 \right)dt
								+ \sigma d\til{X}(t)	\\
				\Rightarrow S(T)	&= S(t)\exp \left[ \left( r + \frac{1}{2}\sigma^2 \right)(T - t)
				 	+ \sigma \sqrt{T - t} \phi \right]	\\
					\phi &\sim N(0, 1)
			\end{aligned}
		\end{equation*}
		Plugging the result of equation above, we can evaluate call option value
		at time $t$ as follows.
		%% CALL OPTION VALUE
		\begin{equation*}
			\begin{aligned}
				V(t)	&= S(t) \E_t^Q \left[ \max \left(1 - \frac{K}{S(T)}, 0 \right) \right]	\\
						&= S(t) \E_t^Q
				\left[ \max \left(1 - \frac{K}{S(t)\exp [ ( r + \frac{1}{2}\sigma^2 )(T - t)
				+ \sigma \sqrt{T - t} \phi ]}, 0 \right) \right]	\\
			\end{aligned}
		\end{equation*}
		Since $1 - \frac{K}{S(t)\exp [ ( r + \frac{1}{2}\sigma^2 )(T - t)
		+ \sigma \sqrt{T - t} \phi ]} \geq 0$ is equivalent to
		$\phi \geq \frac{\log(K / S(t) - (r + \frac{1}{2}\sigma^2)(T - t)}
			{\sigma \sqrt{T - t}} \equiv L$, and $\phi$ follows standard
		normal distribution, call option value can be calculated as follows.
		%% CALL OPTION VALUE 2
		\begin{equation*}
			\begin{aligned}
				V(t)	&= S(t) \int_L^{\infty} {
					\left( 1 - \frac{K}{S(t)
					\exp[(r + \frac{1}{2}\sigma^2)(T - t) + \sigma \sqrt{T - t}x]} \right)
				} {
					\frac{1}{\sqrt{2\pi}} e^{\frac{1}{2}x^2} dx
				}	\\
					&= S(t) \int_L^{\infty}
						\frac{1}{\sqrt{2\pi}} e^{ \frac{1}{2}x^2} dx
					- K \int_L^{\infty}
						\frac{1}{\sqrt{2\pi}} e^{-(r + \frac{1}{2}\sigma^2)(T - t)
						- \sigma \sqrt{T - t}x - \frac{1}{2}x^2}dx	\\
					&= S(t)(1 - N(L))
						- Ke^{-r(T - t)} \int_L^{\infty}
							\frac{1}{\sqrt{2\pi}} e^{-\frac{1}{2}(x + \sigma \sqrt{T - t})^2}dx	\\
					&= S(t)(1 - N(L))
						- Ke^{-r(T - t)} \int_{L + \sigma \sqrt{T - t}}^{\infty}
							\frac{1}{\sqrt{2\pi}} e^{-\frac{1}{2} y^2}dy	\\
					&= S(t)(1 - N(L)) - Ke^{-r(T - t)}(1 - N(L + \sigma \sqrt{T - t}))	\\
					&= S(t)N(-L) - Ke^{-r(T - t)}N(-L - \sigma \sqrt{T - t})	\\
					&= S(t)N(d_1) - Ke^{-r(T - t)}N(d_2)	\\
			\end{aligned}
		\end{equation*}
		\begin{equation*}
			\begin{aligned}
				y 	&= x + \sigma \sqrt{T - t}	\\
				d_1 &= \frac{\log(S_t / K) + (r + \frac{1}{2}\sigma^2)(T - t)}{\sigma \sqrt{T - t}}	\\
				d_2 &= d_1 - \sigma \sqrt{T -t}
			\end{aligned}
		\end{equation*}
		Which is the desired solution. It is not easier than when $B(t)$ was
		numeraire because it also has some integration and transformation in
		evaluating procedure.
	\end{enumerate}
\end{homeworkProblem}

%----------------------------------------------------------------------------------------
%	PROBLEM 3
%----------------------------------------------------------------------------------------
\begin{homeworkProblem}
	\begin{enumerate}[(a)]
		\item	%% (a)
		Using Ito's lemma, dynamics of $\log B_e$ is as follows.
		%% DYNAMICS OF LOG(Be)
		\begin{equation*}
			\begin{aligned}
				d\log B_e	&= \frac{1}{B_e}dB_e	\\
							&= \frac{1}{B_e}r_e B_e dt \\
							&= r_e dt
			\end{aligned}
		\end{equation*}
		Since $B_e(0) = 0$, $B_e(T)$ is evaluated as $B_e(0)e^{r_e T} = e^{r_e T}$.
		\item	%% (b)
		By Ito's product rule, dynamics of $Z_e$ and $Z_g$ is derived as
		follows.
		%% DYNAMICS OF Ze
		\begin{equation*}
			\begin{aligned}
				dZ_e	&= d(eB_e) = B_ede + edB_e + dB_ede	\\
						&= B_e(\mu_e edt + \sigma_e e dX_2) + er_eB_edt	\\
						&= (\mu_e + r_e)Z_edt + \sigma_e Z_e dX_2	\\
				dZ_g	&= d(gB_g) = B_gdg + gdB_g + dB_gdg	\\
						&= B_g(\mu_g gdt + \sigma_g g dX_1) + gr_gB_gdt	\\
						&= (\mu_g + r_g)Z_gdt + \sigma_g Z_g dX_1
			\end{aligned}
		\end{equation*}
		\item	%% (c)
		By Ito's product rule, dynamics of $X$ is derived as follows.
		%% DYNAMICS OF X
		\begin{equation*}
			\begin{aligned}
				dX	&= d\left( \frac{g}{e} \right)
						= \frac{1}{e}dg + gd\left( \frac{1}{e} \right) + dgd\left( \frac{1}{e} \right)	\\
				d\left( \frac{1}{e} \right)
					&= -\frac{1}{e^2}de + 2 \times \frac{1}{2} \times \frac{1}{e^3}(de)^2	\\
					&= -\frac{1}{e^2}(\mu_e e dt + \sigma_e e dX_2)
						+ \frac{1}{e^3}\sigma_e^2 e^2 dt	\\
					&= (-\mu_e + \sigma_e^2) \frac{1}{e} dt - \sigma_e \frac{1}{e}dX_2	\\
				\Rightarrow dX	&= \frac{1}{e}(\mu_g g dt + \sigma_g g dX_1)
					+ g[(-\mu_e + \sigma_e^2)\frac{1}{e}dt - \sigma_e \frac{1}{e} dX_2]
					+ (\mu_g g dt + \sigma_g g dX_1)[(-\mu_e + \sigma_e^2)\frac{1}{e}dt - \sigma_e \frac{1}{e} dX_2]	\\
								&= (\mu_g - \mu_e + \sigma_e^2 - \rho \sigma_e \sigma_g)Xdt
					+ X(\sigma_g dX_1 - \sigma_e dX_2)
			\end{aligned}
		\end{equation*}
		Let $dX_3 = \frac{\sigma_g dX_1 - \sigma_e dX_2}{\sqrt{\sigma_g^2 - 2 \sigma_g \sigma_e + \sigma_e^2}}$.
		Then by Levy's theorem, $X_3$ is a brownian motion.
		Let $\sigma_X^2 = \sigma_g^2 - 2 \sigma_g \sigma_e + \sigma_e^2$,
		then $dX$ can be represented as follows.
		%% DYNAMICS OF X - 2
		\begin{equation*}
			\begin{aligned}
				dX	&= (\mu_g - \mu_e + \sigma_e^2 - \rho \sigma_e \sigma_g)Xdt
					+ \sigma_X X dX_3
			\end{aligned}
		\end{equation*}
		\item	%% (d)
		By Ito's product rule, dynamics of $Y$ is derived as follows.
		%% DYNAMICS OF Y
		\begin{equation*}
			\begin{aligned}
				dY 	&= d\left( \frac{Z_g}{Z_e} \right)
					= Z_g d\left( \frac{1}{Z_e} \right) + \frac{1}{Z_e}dZ_g
						+ dZ_g d\left( \frac{1}{Z_e} \right)	\\
				d\left( \frac{1}{Z_e} \right) &=
					-\frac{1}{Z_e^2}dZ_e + \frac{1}{Z_e^3}(dZ_e)^2	\\
					&= -\frac{1}{Z_e}[(\mu_e + r_e)Z_edt + \sigma_e Z_e dX_2]
						+ \frac{1}{Z_e^3}\sigma_e^2 Z_e^2 dt	\\
					&= [-(\mu_e + r_e) + \sigma_e^2]\frac{1}{Z_e}dt
						- \sigma_e \frac{1}{Z_e}dX_2	\\
				\Rightarrow dY
					&= \frac{Z_g}{Z_e}([-(\mu_e + r_e) + \sigma_e^2]dt
						- \sigma_e dX_2)	\\
						&~~+ \frac{Z_g}{Z_e}[(\mu_g + r_g)dt + \sigma_g dX_1]	\\
						&~~+ \frac{Z_g}{Z_e}([-(\mu_e + r_e) + \sigma_e^2]dt
							- \sigma_e dX_2)
							[(\mu_g + r_g)dt + \sigma_g dX_1]	\\
					&= [(\mu_g + r_g) - (\mu_e + r_e) + \sigma_e^2 - \rho \sigma_e \sigma_g]Ydt
						+ \sigma_X Y dX_3
			\end{aligned}
		\end{equation*}
		\item	%% (e)
		By fundamental theorem of finance, if market is complete, there exists
		a martingale measure such that $Y$ is a martingale under the measure.
		Therefore, under $\mathbf{Q}$, there is no drift term in $dY$ so that
		$Y$ is a martingale.
		\item	%% (f)
		By Girsanov's theorem, if we choose $\til{X}_3(t)$ such that
		$\til{X}_3(t) = X_3(t) + \int_0^t \frac{(\mu_g + r_g) - (\mu_e + r_e)
			+ \sigma_e^2 - \rho \sigma_e \sigma_g}{\sigma_X} ds$, then
		there exists a probability measure $\mathbf{Q}$ so that
		$\til{X}_3$ is a brownian motion under $\mathbf{Q}$.
		Plugging $dX_3 = d\til{X}_3 - \frac{(\mu_g + r_g) - (\mu_e + r_e)
			+ \sigma_e^2 - \rho \sigma_e \sigma_g}{\sigma_X}dt$
		into $dY$, it becomes to $dY = \sigma_X Y d\til{X}_3$, which is
		a martingale.
		\item	%% (g)
		Under $\mathbf{Q}$, $\til{X}_3(t) = X_3(t) + \int_0^t \frac{(\mu_g + r_g) - (\mu_e + r_e)
			+ \sigma_e^2 - \rho \sigma_e \sigma_g}{\sigma_X} ds$
		is a brownian motion.
		Plugging $dX_3 = d\til{X}_3 - \frac{(\mu_g + r_g) - (\mu_e + r_e)
			+ \sigma_e^2 - \rho \sigma_e \sigma_g}{\sigma_X}dt$
		into $dX$, dynamics of $X$ is derived as follows.
		%% DYNAMICS OF X UNDER Q
		\begin{equation*}
			\begin{aligned}
				dX	&= (\mu_g - \mu_e + \sigma_e^2 - \rho \sigma_e \sigma_g)Xdt
					+ \sigma_X X dX_3	\\
					&= (\mu_g - \mu_e + \sigma_e^2 - \rho \sigma_e \sigma_g)Xdt
						+ \sigma_X X (d\til{X}_3 - \frac{(\mu_g + r_g) - (\mu_e + r_e)
							+ \sigma_e^2 - \rho \sigma_e \sigma_g}{\sigma_X}dt)	\\
					&= (r_e - r_g)Xdt + \sigma_X X d\til{X}_3
			\end{aligned}
		\end{equation*}
		\item	%% (h)
		Let $\eta(t) = \E_t^Q[V(T)/Z_e(T)]$. Then $\eta(t)$ is a $\mathbf{Q}$ -
		martingale. Since $Y = Z_g / Z_e$ is also a martingale under $\mathbf{Q}$,
		by martingale representation theorem, there exists a unique process
		$\phi(t)$ such that $d\eta(t) = \phi(t)dY(t)$. Then, let us construct
		a portfolio $\Pi(t)$ such that $\Pi(t) = \phi(t)Z_g(t) + \psi(t)Z_e(t)$,
		where $\psi(t) = \eta(t) - \phi(t) Z_g(t) / Z_e(t)$ so that $\Pi(t) =
		\eta(t)Z_e(t)$ for all $t$. Since $\Pi(T) = \eta(T)Z_e(T) = V(T)$,
		if the strategy is self-financing, by no arbitrage principle, $V(t)$
		must be equal to $\eta(t)Z_e(t) = Z_e(t) \E_t^Q[V(T)/Z_e(T)]$. Therefore,
		let us figure out whether the strategy is self-fiancing or not by deriving
		dynamics of $\Pi(t)$ as follows.
		\begin{equation*}
			\begin{aligned}
				d\Pi(t)	&= d(\eta(t)Z_e(t))	\\
						&= \eta(t)dZ_e(t) + Z_e(t)d\eta(t) + dZ_e(t)d\eta(t)	\\
						&= \left( \psi(t) + \phi(t)\frac{Z_g(t)}{Z_e(t)} \right)dZ_e(t)
							+ Z_e(t) \left( \phi(t) d\left( \frac{Z_g(t)}{Z_e(t)} \right) \right)
							+ \phi(t) d\left( \frac{Z_g(t)}{Z_e(t)} \right)dZ_e(t)	\\
						&= \phi(t) \left[ \frac{Z_g(t)}{Z_e(t)}dZ_e(t)
						 	+ Z_e(t)d\left( \frac{Z_g(t)}{Z_e(t)} \right)
							+ d\left( \frac{Z_g(t)}{Z_e(t)} \right) dZ_e(t) \right]
							+ \psi(t)dZ_e(t)	\\
						&= \phi(t) d\left( Z_e(t) \frac{Z_g(t)}{Z_e(t)} \right)
							+ \psi(t)dZ_e(t)	\\
						&= \phi(t)dZ_g(t) + \psi(t)dZ_e(t)
			\end{aligned}
		\end{equation*}
		From the equation above, it can be found that $\Pi(t)$ is a self-financing
		portfolio. Therefore, as mentioned above, value of the option $V(t)$
		must be equal to $\Pi(t)$, which is $Z_e(t) \E_t^Q[V(T)/Z_e(T)]$.
		Since $Z_e = eB_e$, and $V(T) = 1000 \times \max[g_T - e_T, 0]$,
		$V(t)$ can also be represented as follows.
		\begin{equation*}
			\begin{aligned}
				V(t)	&= Z_e(t) \E_t^Q \left[\frac{V(T)}{Z_e(T)} \right]	\\
						&= e_t B_e(t) \times 1000 \times
							\max \left[ \frac{g_T - e_T}{e_T B_e(T)}, 0 \right]	\\
						&= e_t e^{-r_e(T - t)} \times 1000 \times
							\max[X(T) - 1, 0]
			\end{aligned}
		\end{equation*}
		\item	%% (i)
		By analogy to the Black-Scholes-Merton formula, $V(g, e, 0)$ is
		represented as follows.
		%% CALL OPTION VALUE
		\begin{equation*}
			\begin{aligned}
				V(0)	&= e_0 e^{-r_eT} \times 1000 \times
					[X_0 e^{(r_e - r_g)T}N(d_1) - N(d_2)]	\\
						&= e_0 e^{-r_eT} \times 1000 \times
					\left[ \left( \frac{g_0}{e_0} \right)
					 e^{(r_e - r_g)T}N(d_1) - N(d_2) \right]	\\
					 	&= 1000 e^{-r_eT} [g_0e^{(r_e - r_g)T}N(d_1) - e_0 N(d_2)]	\\
				d_1	&= \frac{\log X_0 + (r_e - r_g + \frac{1}{2}\sigma_X^2)T}{\sigma_X \sqrt{T}}	\\
					&= \frac{\log(g_0 / e_0) + (r_e - r_g + \frac{1}{2}\sigma_X^2)T}{\sigma_X \sqrt{T}}	\\
				d_2 &= d_1 - \sigma_X \sqrt{T}	\\
				\sigma_X^2 &= \sigma_g^2 - 2 \sigma_g \sigma_e + \sigma_e^2
			\end{aligned}
		\end{equation*}
	\end{enumerate}
\end{homeworkProblem}
\end{document}
