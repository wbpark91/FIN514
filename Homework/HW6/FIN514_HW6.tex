%%%%%%%%%%%%%%%%%%%%%%%%%%%%%%%%%%%%%%%%%
% Structured General Purpose Assignment
% LaTeX Template
%
% This template has been downloaded from:
% http://www.latextemplates.com
%
% Original author:
% Ted Pavlic (http://www.tedpavlic.com)
%
% Note:
% The \lipsum[#] commands throughout this template generate dummy text
% to fill the template out. These commands should all be removed when
% writing assignment content.
%
%%%%%%%%%%%%%%%%%%%%%%%%%%%%%%%%%%%%%%%%%

%----------------------------------------------------------------------------------------
%	PACKAGES AND OTHER DOCUMENT CONFIGURATIONS
%----------------------------------------------------------------------------------------

\documentclass{article}

\usepackage{fancyhdr} % Required for custom headers
\usepackage{lastpage} % Required to determine the last page for the footer
\usepackage{extramarks} % Required for headers and footers
\usepackage{graphicx} % Required to insert images
\usepackage{lipsum} % Used for inserting dummy 'Lorem ipsum' text into the template
\usepackage{enumerate}
\usepackage{booktabs}
\usepackage{amsmath}
\usepackage{amssymb}

% Margins
\topmargin=-0.45in
\evensidemargin=0in
\oddsidemargin=0in
\textwidth=6.5in
\textheight=9.0in
\headsep=0.25in

\linespread{1.5} % Line spacing

% Set up the header and footer
\pagestyle{fancy}
\lhead{\hmwkAuthorName} % Top left header
\chead{\hmwkClass\ (\hmwkTitle)} % Top center header
%%\rhead{\firstxmark}
\rhead{} % Top right header
\lfoot{\lastxmark} % Bottom left footer
\cfoot{} % Bottom center footer
\rfoot{Page\ \thepage\ of\ \pageref{LastPage}} % Bottom right footer
\renewcommand\headrulewidth{0.4pt} % Size of the header rule
\renewcommand\footrulewidth{0.4pt} % Size of the footer rule

\setlength\parindent{0pt} % Removes all indentation from paragraphs

%----------------------------------------------------------------------------------------
%	DOCUMENT STRUCTURE COMMANDS
%	Skip this unless you know what you're doing
%----------------------------------------------------------------------------------------

% Header and footer for when a page split occurs within a problem environment
\newcommand{\enterProblemHeader}[1]{
\nobreak\extramarks{#1}{#1 continued on next page\ldots}\nobreak
\nobreak\extramarks{#1 (continued)}{#1 continued on next page\ldots}\nobreak
}

% Header and footer for when a page split occurs between problem environments
\newcommand{\exitProblemHeader}[1]{
\nobreak\extramarks{#1 (continued)}{#1 continued on next page\ldots}\nobreak
\nobreak\extramarks{#1}{}\nobreak
}

\setcounter{secnumdepth}{0} % Removes default section numbers
\newcounter{homeworkProblemCounter} % Creates a counter to keep track of the number of problems

\newcommand{\homeworkProblemName}{}
\newenvironment{homeworkProblem}[1][Problem \arabic{homeworkProblemCounter}]{ % Makes a new environment called homeworkProblem which takes 1 argument (custom name) but the default is "Problem #"
\stepcounter{homeworkProblemCounter} % Increase counter for number of problems
\renewcommand{\homeworkProblemName}{#1} % Assign \homeworkProblemName the name of the problem
\section{\homeworkProblemName} % Make a section in the document with the custom problem count
\enterProblemHeader{\homeworkProblemName} % Header and footer within the environment
}{
\exitProblemHeader{\homeworkProblemName} % Header and footer after the environment
}

\newcommand{\problemAnswer}[1]{ % Defines the problem answer command with the content as the only argument
\noindent\framebox[\columnwidth][c]{\begin{minipage}{0.98\columnwidth}#1\end{minipage}} % Makes the box around the problem answer and puts the content inside
}

\newcommand{\homeworkSectionName}{}
\newenvironment{homeworkSection}[1]{ % New environment for sections within homework problems, takes 1 argument - the name of the section
\renewcommand{\homeworkSectionName}{#1} % Assign \homeworkSectionName to the name of the section from the environment argument
\subsection{\homeworkSectionName} % Make a subsection with the custom name of the subsection
\enterProblemHeader{\homeworkProblemName\ [\homeworkSectionName]} % Header and footer within the environment
}{
\enterProblemHeader{\homeworkProblemName} % Header and footer after the environment
}

%----------------------------------------------------------------------------------------
%	PARTIAL DERIVATIVES
%----------------------------------------------------------------------------------------
\newcommand{\pdv}[3][]{
	\frac{\partial^{#1}{#2}}{\partial{{#3}^{#1}}}
}

%----------------------------------------------------------------------------------------
%	EXPECTATION AND VARIANCE OPERATOR
%----------------------------------------------------------------------------------------
 \newcommand{\E}{\mathrm{E}}
 \newcommand{\Var}{\mathrm{Var}}
 \newcommand{\Cov}{\mathrm{Cov}}
 \newcommand{\Corr}{\mathrm{Corr}}

%----------------------------------------------------------------------------------------
%	ITO's LEMMA
%----------------------------------------------------------------------------------------
\newcommand{\itox}[2]{ \pdv{#1}{#2}d#2 + \frac{1}{2}\pdv[2]{#1}{#2}(d#2)^2}
\newcommand{\itotx}[2]{\pdv{#1}{t}dt + \pdv{#1}{#2}d#2 + \frac{1}{2}\pdv[2]{#1}{#2}(d#2)^2}

%----------------------------------------------------------------------------------------
%	BROWNIAN MOTION
%----------------------------------------------------------------------------------------
\newcommand{\til}[1]{\widetilde{#1}}

%----------------------------------------------------------------------------------------
%	NAME AND CLASS SECTION
%----------------------------------------------------------------------------------------

\newcommand{\hmwkTitle}{Problem Set\ \#6} % Assignment title
\newcommand{\hmwkDueDate}{Wednesday,\ April\ 25,\ 2018} % Due date
\newcommand{\hmwkClass}{FIN\ 514} % Course/class
\newcommand{\hmwkClassTime}{9:30am} % Class/lecture time
\newcommand{\hmwkAuthorName}{Wanbae Park} % Your name

%----------------------------------------------------------------------------------------
%	TITLE PAGE
%----------------------------------------------------------------------------------------

\title{
\vspace{2in}
\textmd{\textbf{\hmwkClass:\ \hmwkTitle}}\\
\normalsize\vspace{0.1in}\small{Due\ on\ \hmwkDueDate}\\
\vspace{3in}
}

\author{\textbf{\hmwkAuthorName}}
\date{} % Insert date here if you want it to appear below your name

%----------------------------------------------------------------------------------------

\begin{document}

\maketitle

%----------------------------------------------------------------------------------------
%	TABLE OF CONTENTS
%----------------------------------------------------------------------------------------

%\setcounter{tocdepth}{1} % Uncomment this line if you don't want subsections listed in the ToC

%%\newpage
%%\tableofcontents
\newpage

%----------------------------------------------------------------------------------------
%	PROBLEM 1
%----------------------------------------------------------------------------------------

% To have just one problem per page, simply put a \clearpage after each problem

\begin{homeworkProblem}
	\begin{enumerate}[(a)]
		\item	%% (a)
		By Ito's product rule, $dY(t)$ satisfies the following equation.
		%% EQUATION: DYNAMICS OF Y(t)
		\begin{equation*}
			\begin{aligned}
				dY(t)	&= B_P(t)dS(t) + S(t)dB_P(t) + dB_P(t)dS(t)	\\
						&= B_P(t)[\mu S(t)dt + \sigma S(t)dX(t)] + r_PS(t)B_P(t)dt	\\
						&= (\mu + r_P)Y(t)dt + \sigma Y(t)dX(t)
			\end{aligned}
		\end{equation*}
		In order to find martingale measure with respect to $B(t)$ as a numeraire,
		dynamics of $Y(t) / B(t)$ is derived as follows.
		%% EQUATION: DYNAMICS OF Y(t) / B(t)
		\begin{equation*}
			\begin{aligned}
				d \left( \frac{Y(t)}{B(t)} \right)
					&= Y(t) d\left( \frac{1}{B(t)} \right)
						+ \frac{1}{B(t)}dY(t) + dY(t) d \left( \frac{1}{B(t)} \right)	\\
				d \left( \frac{1}{B(t)} \right)
					&= -\frac{1}{B^2(t)}dB(t)	\\
					&= -\frac{1}{B^2(t)}rB(t)dt = -r \frac{1}{B(t)}dt	\\
				\Rightarrow d \left( \frac{Y(t)}{B(t)} \right)
					&= Y(t) \left( -r \frac{1}{B(t)}dt \right)
						+ \frac{1}{B(t)} [(\mu + r_P)Y(t)dt + \sigma Y(t)dX(t)]	\\
					&= (\mu + r_P - r) \frac{Y(t)}{B(t)}dt + \sigma \frac{Y(t)}{B(t)}dX(t)
			\end{aligned}
		\end{equation*}
		By Girsanov's theorem, there exists a probability measure such that
		$\til{X}(t) = X(t) + \int_0^t \frac{\mu + r_P - r}{\sigma} ds$ is a
		brownian motion under the measure. Therefore, by plugging
		$dX(t) = d\til{X}(t) - \frac{\mu + r_P - r}{\sigma}dt$ into the equation
		above, then $d\left( \frac{Y(t)}{B(t)} \right)$ becomes $\sigma \frac{Y(t)}{B(t)}d\til{X}(t)$,
		hence becomes martingale because there is no drift.
		Therefore, from the perspective of U.S dollar investor,
		under risk-neutral measure,
		$dX(t) = d\til{X}(t) - \frac{\mu + r_P - r}{\sigma}dt$.
		By plugging it into dynamics of $Y(t)$, we can find dynamics of the U.S
		price of a GBP bond under risk-neutral measure as follows.
		%% EQUATION: DYNAMICS of Y(t) UNDER Q
		\begin{equation*}
			\begin{aligned}
				dY(t)	&= (\mu + r_P)Y(t)dt + \sigma Y(t)dX(t)	\\
						&= (\mu + r_P)Y(t)dt
							+ \sigma Y(t)\left[ d\til{X}(t) - \frac{\mu + r_P - r}{\sigma}dt \right]	\\
						&= rY(t)dt + \sigma Y(t) d\til{X}(t)
			\end{aligned}
		\end{equation*}
		And it is consistent with the fact that expected return of every tradable
		asset is risk-free rate under risk-neutral measure.
		\item	%% (b)
		By plugging $dX(t) = d\til{X}(t) - \frac{\mu + r_P - r}{\sigma}dt$
		into dynamics of $S(t)$, we can find dynamics of U.S. dollar price
		of a British pound under risk-neutral probability as follows.
		%% DYNAMICS OF S(t) UNDER Q
		\begin{equation*}
			\begin{aligned}
				dS(t)	&= \mu S(t)dt + \sigma S(t) dX(t)	\\
						&= \mu S(t)dt + \sigma S(t) \left[ d\til{X}(t) - \frac{\mu + r_P - r}{\sigma}dt \right]	\\
						&= (r - r_P)S(t)dt + \sigma S(t) d\til{X}(t)
			\end{aligned}
		\end{equation*}
		\item	%% (c)
		Unlike the assumption of ordinary Black-Scholes-Merton formula,
		since expected return of underlying asset has changed from $r$ to
		$r - r_P$, formula for call option should be changed to following
		equation.
		%% EQUATION: FOREX OPTION FORMULA
		\begin{equation*}
			\begin{aligned}
				&e^{-rT}[S_0 e^{(r - r_P)T}N(d_1) - KN(d_2)]	\\
				&d_1 = \frac{\log(S_0 / K) + (r - r_P + \frac{1}{2} \sigma^2)T}{\sigma \sqrt{T}}	\\
				&d_2 = d_1 - \sigma \sqrt{T}
			\end{aligned}
		\end{equation*}
	\end{enumerate}
\end{homeworkProblem}

%----------------------------------------------------------------------------------------
%	PROBLEM 2
%----------------------------------------------------------------------------------------
\begin{homeworkProblem}
	\begin{enumerate}[(a)]
		\item	%% (a)
		By Ito's product rule, dynamics of $B(t) / S(t)$ is as follows.
		%% EQUATION: DYNAMICS OF B(t) / S(t)
		\begin{equation*}
			\begin{aligned}
				d \left( \frac{B(t)}{S(t)} \right)
					&= d\left( \frac{1}{S(t)} \right)B(t)
						+ \frac{1}{S(t)}dB(t)
						+ d\left(\frac{1}{S(t)} \right)dB(t)	\\
				d \left( \frac{1}{S(t)} \right)
					&= -\frac{1}{S^2(t)}dS(t) + \frac{1}{2} \times 2 \times \frac{1}{S^3(t)}(dS(t))^2	\\
					&= -\frac{1}{S^2(t)}[(\mu - d)S(t)dt + \sigma S(t)dX(t)]
						+ \frac{1}{S^3(t)} \sigma^2 S^2(t) dt	\\
					&= [-(\mu - d) + \sigma^2] \frac{1}{S^2(t)}dt
						- \sigma \frac{1}{S(t)}dX(t)	\\
				\Rightarrow d \left( \frac{B(t)}{S(t)} \right)
					&= \frac{B(t)}{S(t)}[-(\mu - d) + \sigma^2]dt
						- \sigma \frac{B(t)}{S(t)} dX(t)
						+ r \frac{B(t)}{S(t)} dt	\\
					&= [r - (\mu - d) + \sigma^2] \frac{B(t)}{S(t)} dt
						- \sigma \frac{B(t)}{S(t)} dX(t)
			\end{aligned}
		\end{equation*}
		\item	%% (b)
		By Girsanov's theorem, there exists a probability measure such that
		$\til{X}(t) = X(t) - \int_0^t \frac{r - (\mu - d) + \sigma^2}{\sigma} ds$
		is a brownian motion under the measure. Under the measure, process
		$d \left( \frac{B(t)}{S(t)} \right)$ changes as follows.
		%% DYNAMICS OF B(t)/S(t) UNDER Q
		\begin{equation*}
			\begin{aligned}
				d \left( \frac{B(t)}{S(t)} \right)
					&= [r - (\mu - d) + \sigma^2] \frac{B(t)}{S(t)} dt
						- \sigma \frac{B(t)}{S(t)} dX(t)	\\
					&= [r - (\mu - d) + \sigma^2] \frac{B(t)}{S(t)} dt
						- \sigma \frac{B(t)}{S(t)} \left[d\til{X}(t)
						+ \frac{r - (\mu - d) + \sigma^2}{\sigma} dt \right]	\\
					&= - \sigma \frac{B(t)}{S(t)} d\til{X}(t)
			\end{aligned}
		\end{equation*}
		Therefore, under the measure, $\frac{B(t)}{S(t)}$ is a martingale
		since there is no drift term in dynamics.
		Plugging $dX(t) = d\til{X}(t) + \frac{r - (\mu - d) + \sigma^2}{\sigma} dt$
		into the process of $S(t)$, dynamics of $S(t)$ under martingale measure
		with respect to $S(t)$ as a numeraire is as follows.
		%% DYNAMICS OF S(t) UNDER Q
		\begin{equation*}
			\begin{aligned}
				dS(t)	&= (\mu - d)S(t)dt + \sigma S(t)dX(t)	\\
						&= (\mu - d)S(t)dt
							+ \sigma S(t)\left[ d\til{X}(t) + \frac{r - (\mu - d)
							+ \sigma^2}{\sigma} dt \right]	\\
						&= (r + \sigma^2)S(t)dt + \sigma S(t) d\til{X}(t)
			\end{aligned}
		\end{equation*}
		\item	%% (c)
		Under the martingale measure with respect to $S(t)$ as a numeraire,
		$V(t) / S(t)$ is also a martingale. Therefore, by definition of
		martingale, European call option value $V(t)$ is derived as follows.
		%% EUROPEAN CALL OPTION VALUE UNDER Q
		\begin{equation*}
			\begin{aligned}
				\frac{V(t)}{S(t)} &= \E_t^Q \left[ \frac{V(T)}{S(T)} \right]	\\
				\Rightarrow V(t) &= S(t) \E_t^Q \left[ \frac{V(T)}{S(T)} \right]	\\
					&= S(t) \E_t^Q \left[ \frac{\max(S(T) - K, 0)}{S(T)} \right]	\\
					&= S(t) \E_t^Q \left[ \max \left(1 - \frac{K}{S(T)}, 0 \right) \right]
			\end{aligned}
		\end{equation*}
		Where $K$ is strike price of the option, and $Q$ is a probability
		measure in which $V(t)/S(t)$ is a martingale.
		\item	%% (d)
	\end{enumerate}
\end{homeworkProblem}

%----------------------------------------------------------------------------------------
%	PROBLEM 3
%----------------------------------------------------------------------------------------
\begin{homeworkProblem}
	\begin{enumerate}[(a)]
		\item	%% (a)
		\item	%% (b)
		\item	%% (c)
		\item	%% (d)
		\item	%% (e)
		\item	%% (f)
		\item	%% (g)
		\item	%% (h)
		\item	%% (i)
	\end{enumerate}
\end{homeworkProblem}
\end{document}
